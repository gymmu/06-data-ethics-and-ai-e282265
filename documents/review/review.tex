\documentclass{article}

\usepackage[ngerman]{babel}
\usepackage[utf8]{inputenc}
\usepackage[T1]{fontenc}
\usepackage{hyperref}
\usepackage{csquotes}

\usepackage[
    backend=biber,
    style=apa,
    sortlocale=de_DE,
    natbib=true,
    url=false,
    doi=false,
    sortcites=true,
    sorting=nyt,
    isbn=false,
    hyperref=true,
    backref=false,
    giveninits=false,
    eprint=false]{biblatex}
\addbibresource{../references/bibliography.bib}

\title{Review des Papers "Ethik im Umgang mit Daten" von Latisha Soares de Carvalho\dots}
\author{Tamara Pfyffer}
\date{\today}

\begin{document}
\maketitle

\abstract{
    Dies ist ein Review der Arbeit zum Thema Ethik im Umgang mit Daten von <Latisha Soares de Carvalho>.
}

\chapter{Review}
Latisha hat sich in dieser Arbeit mit der Fragestellung "Wer trägt die Verantwortung, wenn ein KI-System fehlerhafte oder unethische Entscheidungen trifft?" ausernandergesetzt. Dies ist eine sehr gute Frage die man sich stellen kann in dieser modernen Welt und passt zu der heutigen Situation.

\section{Allgemein}
Latishas Dokument bringt einen informativer und durchdachter Überblick über die ethischen Herausforderungen, die existieren bei der Benutzung von KI. Latisha hat ihr Dokument klar strukturiert und bietet guten Überblick. Die Verwendung von einem Bild gibt dem Dokument noch guten Charakter und dadurch macht es das Dokument spannender.

\section{Fragestellung}
Die Fragestellung ist logisch und wird auch gut bearbeitet. Ihre Einführung zeigt den Leser genau um was es geht. Ihr Beispiele zu Haftungsverpflichtung sind gründlich in sinnvoll. Ihr Fazit fässt nochmal alles zusammen und betont die Hauptpunkte und die Verantwortlichkeiten für den Umgang mit KI-Systeme.

\section{Schreibstil}
Ihr Schreibstil ist klar und ihre Struktur des Dokument ist logisch, mit den Unterkapitel. Sie hat gute Beispiele verwendet und dadurch kann man alles besser nachvollziehen.

\section{Verbesserungsvorschläge}
Um diese Arbeit zu verbessern könnte man aktuelle Themen und Beispiele einfügen.

\printbibliography

\end{document}
