\documentclass{article}

\usepackage[ngerman]{babel}
\usepackage[utf8]{inputenc}
\usepackage[T1]{fontenc}
\usepackage{hyperref}
\usepackage{csquotes}

\usepackage[
    backend=biber,
    style=apa,
    sortlocale=de_DE,
    natbib=true,
    url=false,
    doi=false,
    sortcites=true,
    sorting=nyt,
    isbn=false,
    hyperref=true,
    backref=false,
    giveninits=false,
    eprint=false]{biblatex}
\addbibresource{../references/bibliography.bib}

\title{Notizen zum Projekt Data Ethics}
\author{Tamara Pfyffer}
\date{\today}

\begin{document}
\maketitle

\abstract{
    Dieses Dokument ist eine Sammlung von Notizen zu dem Projekt. Die Struktur innerhalb des
    Projektes ist gleich ausgelegt wie in der Hauptarbeit, somit kann hier einfach geschrieben
    werden, und die Teile die man verwenden möchte, kann man direkt in die Hauptdatei ziehen.
}

\tableofcontents

\section{Was ist KI überhaupt?}
-KI hat Menschliche Fähigkeiten. -Kann sachen Wahrnehmen und verarbeiten. -KI ist oft in unserem Alltag wie für shopping und Werbung. - sammelt daten für gezielte werbung

\section{Wie wird KI trainiert?}
Wie man eine KI trainiert? -man füttert daten an die KI. -trainiert KI durch alle Punkte zugehen und wichtige auszuwählen. -jede neue anfrage macht KI besser bis perfekt ist. -Daten sollte man labeln und kategorisieren. 

\section{Fragestellung}
Welche ethischen Probleme gibt er bei der Nutzung von KI für persöhnliche Daten? 

\section{Einleitung zu meinem Thema}
- Menschen verwenden KI für persönliche Daten - Was für probleme gibt das? - Was macht die KI damit? - Für was benutzt die KI unsere Daten, wo von wir nichts wissen? -Ist es schlau die KI für persöhnliche Daten zu verwenden? 

\section{Notizen}
-Ethische Probleme bei der Nutzung von KI für persöhnliche Daten? -Sind Privatsphäre und Datenschutz. -Was ist aber Datenschutz überhaupt? -Datenschutz ist schutz für persönliche Daten, Namen, Adresse,Telefonnummer, E-Mail Adressen u.s.w. -Schützt vor unerlaubten Zugriff und Missbrauch -sicherstellen das daten vertraulich behandelt werden und nur für den bestimmten zweck.  -negativen Ansätze, KI kann grosse Mengen an persöhnlichen Daten analysieren - wird benutzt für profiling und verletzt privatsphäre. -können missbraucht werden, für werbung. -Sicherheit ist nur begrenzt.
-Auch die Sicherheit und Missbrauch sind Probleme. - KI-Systeme sind anfällig für Hacking was zu Missbrauch führen kann. -Missbrauch kommt in Form von Identitätsdiebstahl und Betrug.
- keine Kontrolle über die KI, wir wissen nicht welche und wie er die Daten verarbeitet.
- sollte wissen, bevor man KI für persöhnliche Daten verwendet, dass immer die Gefahr ist, dass sie Missbraucht werden und eine Verletzung von Privatsphäre und Datenschutz passieren könnte. 


\section{Quellen}
\url {Clickworker: https://www.clickworker.de/kunden-blog/kuenstliche-intelligenz-systeme-trainieren/}
\url {https://www.europarl.europa.eu/topics/de/article/20200827STO85804/was-ist-kunstliche-intelligenz-und-wie-wird-sie-genutzt}
\url{https://digitales-institut.de/ki-und-datenschutz-herausforderungen-und-loesungsansaetze/#:~:text=Ein%20zentrales%20Anliegen%20bei%20der%20Nutzung%20von%20KI,Daten%20angemessen%20gesch%C3%BCtzt%20werden%2C%20um%20Missbrauch%20zu%20verhindern.}
\end{document}
