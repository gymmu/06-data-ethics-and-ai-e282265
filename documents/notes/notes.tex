\documentclass{article}

\usepackage[ngerman]{babel}
\usepackage[utf8]{inputenc}
\usepackage[T1]{fontenc}
\usepackage{hyperref}
\usepackage{csquotes}

\usepackage[
    backend=biber,
    style=apa,
    sortlocale=de_DE,
    natbib=true,
    url=false,
    doi=false,
    sortcites=true,
    sorting=nyt,
    isbn=false,
    hyperref=true,
    backref=false,
    giveninits=false,
    eprint=false]{biblatex}
\addbibresource{../references/bibliography.bib}

\title{Notizen zum Projekt Data Ethics}
\author{Tamara Pfyffer}
\date{\today}

\begin{document}
\maketitle

\abstract{
    Dieses Dokument ist eine Sammlung von Notizen zu dem Projekt. Die Struktur innerhalb des
    Projektes ist gleich ausgelegt wie in der Hauptarbeit, somit kann hier einfach geschrieben
    werden, und die Teile die man verwenden möchte, kann man direkt in die Hauptdatei ziehen.
}

\tableofcontents
\section{Einleitung}
Wie man eine KI trainiert? Als erster Schritt muss man die KI mit Daten füttern. das trainiert die KI bei jeder Frage alle Punkte durchzugehen und dann die wichtigen Punkte auszuwählen. Bei jeder neuen Anfrage wird die KI besser bis es keinen Raum für Verbesserung gibt. Man sollte die Daten, welche man der KI verspeisst kategorisieren und labeln.

\section{Fragestellung}
Welche ethischen Probleme gibt er bei der Nutzung von KI für persöhnliche Daten? 

\section{Notizen}
Ethische Probleme bei der Nutzung von KI für persöhnliche Daten?
Sind Privatsphäre und Datenschutz. KI kann zum Beispiel grosse Mengen an persöhnlichen Daten analysieren, was zu detailliertem Profiling führen kann und verletzt die Privatsphäre des Nutzers. Ebenfalls können die Daten für Zwecke benutzt werden, welche der Nutzer nicht zugestimmt hat, wie gezielte Werbung. Ebenfalls ist die Sicherheit für diese Daten nur begrenzt und kann zu missbrauch führen.
Auch die Sicherheit und Missbrauch sind grosse Probleme. KI-Systeme sind anfällig für Hacking was zu Missbrauch von persöhnlichen Daten führen kann. Solcher Missbrauch kommt in der Form von Identitätsdiebstahl und Betrug.
Wir haben auch keine Kontrolle über die KI, wir wissen nicht welche und wie er die Daten verarbeitet.

\section{Quellen}
Clickworker: https://www.clickworker.de/kunden-blog/kuenstliche-intelligenz-systeme-trainieren/
\printbibliography

\end{document}
