\documentclass{report}

\usepackage[ngerman]{babel}
\usepackage[utf8]{inputenc}
\usepackage[T1]{fontenc}
\usepackage{hyperref}
\usepackage{csquotes}
\usepackage[a4paper]{geometry}

\usepackage[
    backend=biber,
    style=apa,
    sortlocale=de_DE,
    natbib=true,
    url=false,
    doi=false,
    sortcites=true,
    sorting=nyt,
    isbn=false,
    hyperref=true,
    backref=false,
    giveninits=false,
    eprint=false]{biblatex}
\addbibresource{../references/bibliography.bib}


\title{Ethik im Umgang mit Daten}
\author{Tamara Pfyffer}
\date{\today}


\begin{document}

\maketitle

\abstract{
    Dies ist eine Vorlage für eine Maturarbeit in der Informatik am Gymnasium Muttenz. Sie dient dazu, die Arbeit schnell und einfach zu starten und sollte einen guten Überblick über die Arbeit bieten.
}

\tableofcontents

\chapter{Was ist KI überhaupt?}
Die Künstliche Intelligenz ist eine Maschine mit den Fähigkeiten von Menschen wie logisches Denken, Lernen, Planen und Kreativität. Die KI kann
Dinge wahrnehmen mit ihnen umgehen und so Probleme lösen und ein bestimmtes Ziel erreichen. Viele Wissen gar nicht das die KI oft versteckt,
in unserem Alltag, genutzt wird. Wie zum Beispiel beim Online-Shopping
und Werbung. Die KI generiert mit gesammelten Daten gezielte Werbung
die uns vielleicht gefallen könnte und für noch vieles mehr wird die KI in
unserem Alltag benutzt.

\chapter{Wie wird KI trainiert?}
Wie man eine KI trainiert? Als erster Schritt muss man die KI mit Daten
füttern. Das trainiert die KI bei jeder Frage alle Punkte durchzugehen und
dann die wichtigen Punkte auszuwählen. Bei jeder neuen Anfrage wird die
KI besser, bis es keinen Raum für Verbesserung gibt. Man sollte die Daten,
welche man der KI verspeist kategorisieren und labeln.



\input{chap_methode.tex}

\section{Etwas mit Quellen}

Etwas mit Änderung hier am Ende.

Wenn ich eine Quelle zitieren möchte, kann ich das ganze einfach am Ende des Satzes machen \citep{example}. Oder wie \citet{example} sagt, auch mitten im Text.

\printbibliography

\end{document}
