\documentclass{report}

\usepackage[ngerman]{babel}
\usepackage[utf8]{inputenc}
\usepackage[T1]{fontenc}
\usepackage{hyperref}
\usepackage{csquotes}
\usepackage[a4paper]{geometry}

\usepackage[
    backend=biber,
    style=apa,
    sortlocale=de_DE,
    natbib=true,
    url=false,
    doi=false,
    sortcites=true,
    sorting=nyt,
    isbn=false,
    hyperref=true,
    backref=false,
    giveninits=false,
    eprint=false]{biblatex}
\addbibresource{../references/bibliography.bib}


\title{Ethik im Umgang mit Daten}
\author{Tamara Pfyffer}
\date{\today}


\begin{document}

\maketitle

\abstract{
    Dies ist eine Vorlage für eine Maturarbeit in der Informatik am Gymnasium Muttenz. Sie dient dazu, die Arbeit schnell und einfach zu starten und sollte einen guten Überblick über die Arbeit bieten.
}

\tableofcontents

\chapter{Künstliche Intelligenz}

\section{Was ist KI überhaupt?}
Die Künstliche Intelligenz ist eine Maschine mit den Fähigkeiten von Menschen wie logisches Denken, Lernen, Planen und Kreativität. Die KI kann
Dinge wahrnehmen mit ihnen umgehen und so Probleme lösen und ein bestimmtes Ziel erreichen. Viele Wissen gar nicht das die KI oft versteckt,
in unserem Alltag, genutzt wird. Wie zum Beispiel beim Online-Shopping
und Werbung. Die KI generiert mit gesammelten Daten gezielte Werbung
die uns vielleicht gefallen könnte und für noch vieles mehr wird die KI in
unserem Alltag benutzt.

\section{Wie wird KI trainiert?}
Wie man eine KI trainiert? Als erster Schritt muss man die KI mit Daten
füttern. Das trainiert die KI bei jeder Frage alle Punkte durchzugehen und
dann die wichtigen Punkte auszuwählen. Bei jeder neuen Anfrage wird die
KI besser, bis es keinen Raum für Verbesserung gibt. Man sollte die Daten,
welche man der KI verspeist kategorisieren und labeln.

\chapter{Frage}

\section{Fragestellung}
Welche ethischen Probleme gibt er bei der Nutzung von KI für persönliche Daten?

\section{Einleitung zu meinem Thema}
In der heutigen modernen Welt gibt es viele Personen welche die KI auch
für persönliche Daten verwenden. Nun stellt sich die Frage was für ethische Probleme es gibt bei der Nutzung von KI für persönliche Daten. Was
macht die KI damit? Für was benutzt die KI unsere Daten, wo von wir
nichts wissen? Ist es schlau die KI für persönliche Daten zu verwenden?

\section{Antwort zu meiner Frage}
Ethische Probleme bei der Nutzung von KI für persönliche Daten? Sind
Privatsphäre und Datenschutz. Was ist aber Datenschutz überhaupt? Datenschutz umfasst der Schutz von persönlichen Daten wie Namen, Adressen, Telefonnummern, E-Mail-Adressen und so weiter. Der Datenschutz
schützt diese Daten vor unerlaubten Zugriff und Missbrauch. Ebenfalls
geht es darum sicherzustellen, dass Informationen vertraulich behandelt
werden und nur für den vorhergesehenen Zweck benutzt werden. Die negativen Ansätze sind jedoch, KI kann zum Beispiel grosse Mengen an persönlichen Daten analysieren, was zu detailliertem Profling führen kann
und verletzt die Privatsphäre des Nutzers. Ebenfalls können die Daten für
Zwecke benutzt werden, welche der Nutzer nicht zugestimmt hat, wie gezielte Werbung. Ebenfalls ist die Sicherheit für diese Daten nur begrenzt
und kann zu missbrauch führen. Auch die Sicherheit und Missbrauch sind
grosse Probleme. KI-Systeme sind anfällig für Hacking was zu Missbrauch
von persönlichen Daten führen kann. Solcher Missbrauch kommt in der
Form von Identitätsdiebstahl und Betrug. Wir haben auch keine Kontrolle über die KI, wir wissen nicht welche und wie er die Daten verarbeitet.
Man sollte unbedingt wissen, bevor man die KI für persönliche Daten
verwendet, dass sobald man der KI diese Daten verfüttert kann immer die
Gefahr sein, dass sie missbraucht werden und eine Verletzung von Privatsphäre und Datenschutz passieren könnte.

\chapter{Schlusswort und Quellen}

\section{Schlusswort}
Um kurz zusammenzufassen, die KI bringt Vorteile und Nachteile bei der Benutzung der KI für persönliche Daten. Die KI versichert jedoch keine komplette Sicherheit. Darum sollte man aufpassen und es lieber vermeiden die KI für solche Daten zu verwenden.

\section{Quellen}
\url {Clickworker: https://www.clickworker.de/kunden-blog/kuenstliche-intelligenz-systeme-trainieren/}
\url {https://www.europarl.europa.eu/topics/de/article/20200827STO85804/was-ist-kunstliche-intelligenz-und-wie-wird-sie-genutzt}
\url{https://digitales-institut.de/ki-und-datenschutz-herausforderungen-und-loesungsansaetze/#:~:text=Ein%20zentrales%20Anliegen%20bei%20der%20Nutzung%20von%20KI,Daten%20angemessen%20gesch%C3%BCtzt%20werden%2C%20um%20Missbrauch%20zu%20verhindern.}
\end{document}


\input{chap_methode.tex}

\section{Etwas mit Quellen}

Etwas mit Änderung hier am Ende.

Wenn ich eine Quelle zitieren möchte, kann ich das ganze einfach am Ende des Satzes machen \citep{example}. Oder wie \citet{example} sagt, auch mitten im Text.

\printbibliography

\end{document}
